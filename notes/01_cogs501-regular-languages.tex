\documentclass[11pt]{article}

\usepackage[nohide,twocolumn]{ulecnot}
\usepackage{natbib}


\usepackage{pgf}
\usepackage{tikz}
\usetikzlibrary{arrows,automata}

\usepackage{hyperref}


\pagestyle{fancy}
\lhead{COGS 501 -- Formal Languages and Linguistics}
\chead{Formal Language Basics and Regular Languages}
	\rhead{Updated \it \today }
\lfoot{Umut \"Ozge}
\cfoot{}
\rfoot{Page \thepage/\pageref{LastPage}}
\setlength{\headheight}{13.6pt}

\newcommand{\emptystring}{\ensuremath{\epsilon}}


\begin{document}
\tableofcontents

\section{Alphabets,  strings, languages}

\ezimeti{

\item An \uterm{alphabet} is a finite set of \uterm{symbols}.

\item[] E.g.\ the Roman alphabet $\{a,\ldots,z\}$, $\{a,b\}$, $\crbr{0,1}$,\ldots

\item A \uterm{string} over an alphabet is a finite sequence of symbols from that alphabet.

\item The \uterm{empty string} consists of zero symbols. We will denote it by the symbol
`\emptystring'.
% %\footnote{Sudkamp uses `$\lambda$'  for the empty string, you may 
% also encounter `$e$' or other symbols in other books. As `$\lambda$' has another
% ubiquitous use in computer science and linguistics, I will use `\emptystring'.}

\item[] Examples of strings on alphabet $\{a,b\}$ are \emptystring, $a$,
$abbaba$, \ldots
\item[] The symbols $u,v,w,x,y,z$ are used to name strings -- therefore we avoid
them as symbols of alphabets.

\item The set of all strings -- including \emptystring -- over an alphabet $\Sigma$ is denoted by \sigmastar.

\item The \uterm{length} of a string $w$ is denoted by $|w|$.

\item The \uterm{concatenation} of two strings $w$ and $v$ is formed by
sequencing the strings in the given order; it is denoted as $wv$, $w\circ v$,
or $w^\frown v$. Concatenation is associative: $(xy)z = x(yz)$, and
$\emptystring w = w \emptystring = w$.


\item A string $v$ is a \uterm{substring} of a string $w$, if there exists
strings $x$ and $y$ such that $w=xvy$. Either or both of $x$ and $y$ can be
\emptystring. 

\item The \uterm{reverse} of a string $w$, denoted as $w^R$, is defined as
follows:

\begin{udefinition}[Reverse of a string]
\ezimeti{
\item[]
\item[i.] If $w = \emptystring$, then $w^R = w$.
\item[ii.] If $w = va$ for some $a \in \Sigma$, then $w^R = av^R$.
}
\end{udefinition}

\item The notation $w^n$ stands for concatenating $w$ to itself for $n$ times.
$w^0 = \emptystring$, $w^1 = w$, $w^2 = ww$, and so on.

\item A \uterm{language} is a set of strings over a certain alphabet.

\item Therefore a language $L$ on an alphabet $\Sigma$ is a subset of
\sigmastar.

\item Some example languages over $\Sigma = \{a,b\}$:

\[
\{b,aa,ab\}
\]
\[
\setabs{w\in \sigmastar}{w \text{ has equal number of }a\text{'s and
}b\text{'s}}\\
\]
\[
\setabs{w\in \sigmastar}{w=w^R}\\
\]

\hrulefill
\begin{uexercise}\label{which-is-lang}
Which of the following are languages?
\[
\emptystring\quad\quad\{\emptystring\}\quad\quad\emptyset\quad\quad\Sigma\quad\quad\sigmastar
\]


\end{uexercise}
\hrulefill



}



\section{Some operations on languages}

\ezimeti{ 

\item Given that languages are sets, ordinary set operations
\cttr{union}, \cttr{intersection} and \cttr{difference} are defined
for languages. 

\item There also are operations specific to languages. One is
\cttr{concatenation of languages}. Given any languages $L_1$ and $L_2$
over $\Sigma$, their concatenation, designated as 
$L_1\circ L_2$, $L_1^\frown L_2$, or simply $L_1L_2$, is defined as
follows:
\begin{align}
L_1L_2=\{
w \in \Sigma^*\ |\ 
w = xy\text{, for some } x\in L_1 \text{ and } y\in L_2
\}
\end{align}

\item Our final and third operation is \textbf{closure} (or
\textbf{star}, or \textbf{Kleene closure}) of a language $L$, denoted
as $L^*$, which is the set of expressions formed by concatenating zero or more
strings from $L$.  Formally, 
\[
L^*=\setabs{w\in \sigmastar}{w=w_1w_2\ldots w_k \text{ for some }k\geq 0 \text{ where }w_1,w_2,\ldots w_k \in L}
\]
or,
\begin{align*}
&L^* = \bigcup_{i = 0}^\infty L^i \\ & \text{where }
L^0=\crbr{\epsilon},\ L^1=L,\ \text{ and, } L^i=LL\cdots L\text{,
with } i\text{-many } L\text{s} &
\end{align*}


\item We write $L^+$ in place of $LL^*$, which is:
\[
L^*=\setabs{w\in \sigmastar}{w=w_1w_2\ldots w_k \text{ for some }k\geq 1 \text{ where }w_1,w_2,\ldots w_k \in L}
\]


\hrulefill
\begin{uexercise}\label{extf1}
State whether true or false:
\begin{enumerate}
\item\label{extf1a} $\{\epsilon\}^* = \{\epsilon\}$
\item\label{extf1b} $\emptyset^*=\{\epsilon\}$
\item\label{extf1c} For any alphabet $\Sigma$, any $L$ defined over $\Sigma$ is such that
$L\in \mathcal{P}(\Sigma^*)$. ($\mathcal{P}(X)$ denotes the power set of $X$.)
\item\label{extf1cc}  For any language $L$, $L^* = (L^*)^*$.
\item\label{extf1d} For any language $L$, $\emptyset L = L\emptyset = L$
\item\label{extf1e} For any language $L$, $\{\epsilon\}L = \emptyset$
\end{enumerate}

\hyperlink{extf1sol}{\qed}
\end{uexercise}

% \begin{uexercise}\label{write-ext} Give $L^*$ for (i) $L=\{0,1\}$;  (ii) $L$ is the set of
% 	strings of 0's, and (iii) $L=\emptyset$. 
% \end{uexercise}

\begin{uexercise}\label{exconcuni}
Let $L_1 = \setabs{w \in \{ a,b\}^*}{|w| = 2}$ and $L_2\setabs{w \in \{ a,b\}^*}{|w| = 3 \text{ and } w \text{ ends with } b}$

\begin{enumerate}
\item\label{exconcunia} Give the concatenation of $L_1$ and $L_2$.
\item\label{exconcunib} Give their union.
\end{enumerate}

\hyperlink{exconcunisol}{\qed}
\end{uexercise}



\begin{uexercise}\label{lang-desc}
Let $L_1=\setabs{w \in \crbr{\text{0,1}}^*}{w \text{ has an even number of
0's}}$ and  \\
$L_2= \setabs{w \in \crbr{\text{0,1}}^*}{w \text{ starts
with a 0 followed by any number of 1's}}$. \\ Which language is $L_1L_2$?

\hyperlink{lang-desc-sol}{\qed}
\end{uexercise}
\hrulefill

}
\newpage
\section{Finite representation of languages}

\ezimeti{
\item In the theory of computation and its applications we are interested in
representing languages of our interest with \emph{finite} means. This is easy
when the language is finite, but it is a challenge for nonfinite
languages.

\item One method is constructing an \uterm{inductive} definition:


\begin{uexample}[Inductive definition of a language]
The language $L$ over $\{a,b\}$, where each string begins with an $a$ and has
an even length.
\ezimeti{
\item[i.] $aa$ and $ab \in L$.
\item[ii.] If $w \in L$, then $waa$, $wab$, $wba$, $wbb \in L$.
\item[iii.] Nothing other than the strings obtained via i.\ and ii.\ above are
in $L$.
}
\end{uexample}

\begin{uexercise}\label{induct}
Write an inductive definition for the language $L$ over $\crbr{a,b}$ in which
every occurrence of $b$ is immediately preceded by an $a$.

\hyperlink{induct-sol}{\qed}
\end{uexercise}


\item Now let us see a more transparent and direct way of specifying
the above languages. This method involves applying the operations set union, concatenation and
closure on sets. 

\begin{uexample}
The language $L$ over $\{a,b\}$ which has $bb$ as a substring can be defined as
$\{a,b\}^*\{bb\}\{a,b\}^*$.
\end{uexample}

\begin{uexercise}\label{set-notation}
\begin{enumerate}
\item[]
\item\label{set-not1} Define the language $L$ over $\{a,b\}$ whose strings either start with $aa$ or end with $bb$.
\item Define the language $L$ over $\{a,b\}$ whose strings have an even
length. Also define for odd length.
\item Define the language $L$ over $\{0,1\}$ whose strings have two or three
occurrences of 1 the second and third of which are not consecutive.
\end{enumerate}
\end{uexercise}


}

\section{Regular languages}

\ezimeti{

\item Another central point of interest in the theory of computation is classes
of languages -- the set of all languages that share a certain mathematically specifiable
property.


\item The first class we will look at is the class (or set) of \uterm{regular languages}
(or \uterm{regular sets}).



\begin{udefinition}[Regular Languages]\label{def-reglang}
Given an alphabet $\Sigma$:

\etaremune{
\item $\emptyset$ is a regular language.
\item For any symbol $a \in \Sigma$, $\crbr{a}$ is a regular language.
\item If $A$ and $B$ are regular languages, so is $A\cup B$.
\item If $A$ and $B$ are regular languages, so is $AB$. 
\item If $A$ is a regular language, so is $A^*$.
\item Nothing is a regular language unless it fits the above
definition.
}
\end{udefinition}

\item In other words, a language is regular if it can be constructed
from unit languages like $\{a\}$, $\{b\}$ etc.\  and the empty
language $\emptyset$ by the repeated application of
union, concatenation and closure.

\item Precedence conventions: Kleene star binds most tightly, then comes
concatenation, and finally union. For instance, $\crbr{a}\crbr{b}^*$ gives
$\crbr{a,ab,abb,abbb,\ldots}$; if you want
$\crbr{\epsilon,ab,abab,ababab,\ldots}$, you need to have 
$(\crbr{a}\crbr{b})^*$, Again, $\crbr{a}\crbr{b}\cup\crbr{c}$ gives the set
$\crbr{ab,c}$; if you want to have $\crbr{ab,ac}$ you need
$\crbr{a}(\crbr{b}\cup\crbr{c})$. 


\begin{uexercise}\label{show-reg}
Show that the following languages are regular. 
\etaremune{
\item
$L=\setabs{x \in \crbr{a,b}^*}{x \text{ contains an odd number of $b$'s}}$  
\item
$L=\setabs{x \in \crbr{a,b}^*}{x \text{ contains exactly two or three 
$b$'s}}$
}
\end{uexercise}

\item \textbf{Regular expressions} are notational devices to represent
regular languages.

\begin{udefinition}[Regular Expressions]  For each regular expression
$E$,  the language denoted by it is designated as $L(E)$. The set of
regular expressions can be inductively defined as follows.

\etaremune{
\item The constants $\mathbf{\epsilon}$ and $\mathbf{\emptyset}$ are regular
expressions, where $L(\mathbf{\epsilon})=\{\epsilon\}$ and
$L(\mathbf{\emptyset})=\emptyset$.
\item If $a$ is a symbol, $\mathbf{a}$ is a regular expression, where
$L(\mathbf{a})=\{a\}$.
\item If $E$ and $F$ are regular expressions, so is $E\cup F$, where
$L(E\cup F)=L(E)\cup L(F)$.
\item If $E$ and $F$ are regular expressions, so is $EF$, where
$L(EF)=L(E)L(F)$.
\item If $E$ is a regular expression, so is $E^*$, where
$L(E^*)=L(E)^*$
\item If $E$ is a regular expression, so is $(E)$, where $L((E))=L(E)$
\item If $E$ is a regular expression, then it can be shown to be so by 1--6.
}
\end{udefinition}


\item Same precedence conventions as above apply.

\begin{uexercise}\label{write-re} Write regular expressions for the following languages:
\etaremune{ 
\item The set of strings over alphabet $\{a,b,c\}$ containing at least one $a$ and at least one $b$.
\item The set of strings that consist of alternating 0's and 1's (=strings with
no consecutive 0's or 1's).
\item The set of strings of 0's and 1's whose third symbol from the right end is 1.
\item The set of strings of 0's and 1's with at most one pair of consecutive 1's.
\item The set of strings of 0's and 1's with no substring 111. 
}
\end{uexercise}
}

\newpage
\ 
\newpage
	\appendix
\section{Answers for selected exercises}
\begin{itemize}

\item[\ref{which-is-lang}] 

$\emptystring$ is a string, not a language; $\{\emptystring\}$,
$\emptyset$, $\Sigma$, $\sigmastar$ are all
sets of strings, therefore are all languages. 

\item[\ref{extf1}]\hypertarget{extf1sol}{} \ref{extf1a}: T, \ref{extf1b}: T, \ref{extf1c}: T, \ref{extf1cc}: T,\ref{extf1d}: F, \ref{extf1e}: F

\item[\ref{exconcuni}]\hypertarget{exconcunisol}{}

\ref{exconcunia}.
\begin{align*}
L_1L_2= \crbr{&aaaab,aaabb,aabab,aabbb,abaab,ababb,abbab,abbbb,\\
& baaab,baabb,babab,babbb,bbaab,bbabb,bbbab,bbbbb}
\end{align*}


% \item[\ref{write-ext}] (i) $L^*= \crbr{0,1}^*$; (ii) $L^*=\crbr{0}^*$; (iii)
% $L^*=\crbr{\epsilon}$.

\item[\ref{lang-desc}]\hypertarget{lang-desc-sol}{} Strings with odd number of 0's. 

\item[\ref{induct}]\hypertarget{induct-sol}{}

\ezimeti{
\item[i.] $\epsilon \in L$.
\item[ii.] If $w \in L$, then $wab$ and $wa \in L$.
\item[iii.] Nothing other than the strings obtained via i.\ and ii.\ above are
in $L$.
}

\item[\ref{set-notation}]
\begin{enumerate}
\item  $(\{aa\}\{a,b\}^*)\cup(\{a,b\}^*\{bb\})$;


\item even length:
$\{ab,bb,ba,ab\}^*$ or $(\{a,b\}\{a,b\})^*$;

odd length:
$\{ab,bb,ba,ab\}^*\{a,b\}$ or $(\{a,b\}\{a,b\})^*\{a,b\}$;


\item $\crbr{0}^*\crbr{1}\crbr{0}^*\crbr{1}\crbr{0}^*\crbr{01,\epsilon}\crbr{0}^*$.

\end{enumerate}
\item[\ref{show-reg}] We need to show that the given languages can be defined
according to Definition~\ref{def-reglang}. There are two tricky points. One,
you can only use unit languages in your solution, you are not allowed to
use sets like $\crbr{a,b}$, you need to obtain them by union. Two,  you are not
given the empty string $\epsilon$ in the definition, so you cannot use the unit
language $\crbr{\epsilon}$  in your solution. However, you can obtain it from
the empty set, since $\emptyset^*=\crbr{\epsilon}$.

\begin{enumerate}
\item $\crbr{a}^*\crbr{b}\crbr{a}^*(\crbr{a}^*\crbr{b}\crbr{a}^*\crbr{b}\crbr{a}^*)^*$ 

\item $\crbr{a}^*\crbr{b}\crbr{a}^*\crbr{b}\crbr{a}^*(\crbr{b}\cup\emptyset^*)\crbr{a}^*$ 

\end{enumerate}


\item[\ref{write-re}]
\begin{enumerate}
\item $((a\cup b \cup c)^*a(a\cup b \cup c)^*b(a\cup b \cup c)^*) \cup ((a\cup b \cup c)^*b(a\cup b \cup c)^*a(a\cup b \cup c)^*)$

or $(a\cup b \cup c)^*((a(a\cup b \cup c)^*b) \cup (b(a\cup b \cup c)^*a))(a\cup b \cup c)^*$

\item $(1\cup\epsilon)(01)^*(0\cup\epsilon)$

\item $(0\cup 1)^*1(0\cup 1)(0\cup 1)$

\item $(0\cup 10)^*(11\cup 1\cup \epsilon)(0\cup 01)^*$

\item $(0\cup 10 \cup 110)^*(11\cup 1\cup \epsilon)$

\end{enumerate}
\end{itemize}
\end{document}

