\documentclass[11pt]{article}

\usepackage[nohide,twocolumn]{ulecnot}
\usepackage{tipa}

\usepackage{tikz-qtree}

\pagestyle{fancy}
\lhead{COGS 501 -- Linguistics and Formal Languages}
\chead{Syntax}
\rhead{Spring 2017}
\lfoot{Umut \"Ozge}
\cfoot{\emph{Last updated} \today}
\rfoot{Page \thepage/\pageref{LastPage}}
\setlength{\headheight}{13.6pt}

\usepackage[T1]{fontenc}
\usepackage[utf8]{inputenc}
\usepackage{linguex}
	\renewcommand{\refdash}{}
\usepackage{natbib,unatbib}

\begin{document} \section{Introduction}
\begin{itemize}
\item A grammar is a system of rules and representations that determines
how sound is related to meaning. 

\item Syntax lies at the core of grammar.

\item This specification of sound-meaning relation is usually thought in two
steps; first the syntactically well-formed expressions are specified, then how
these well-formed expressions are to be understood is specified. This type of
characterization is sometimes followed by the misleading statement that only the
first part falls within the domain of syntax. In whichever
way stated, remember that what we mean by grammar and syntax encompasses all the
way from form (sound) to meaning, it is never only the specification of form.

\item  The task of a grammar is not just specifying the grammatical sentences
of a language, it is also required to assign the \emph{right structure} to them, so that it
constitutes an account of what the speakers understand from the sentences. 

\item Consider the following pair of sentences, which look so alike in structure, but are
they so?


\ex. 
\a. I persuaded John to leave.
\b. I expected John to leave.


\item Let's take the following example and see what should a successful grammar
specify about this sentence?

\ex.  It was Bill who John could have persuaded to examine himself.

\begin{uexercise}

Try to specify the rules that govern which expression in the first set goes with
which in the second.


\ex. Ahmet'in eve \altset{	\text{geldiğini}\\ 
							\text{gelmesini} \\ 
							\text{gelişini}} 
							\altset{\text{özlüyorum.}\\
									\text{bekliyorum.}\\
									\text{istiyorum.}\\
									\text{biliyorum.}\\
									\text{umuyorum.}}


\end{uexercise}

% \item \emph{Grammar} versus \emph{syntax}.
% \item \emph{Description} versus \emph{Explanation}.
% \item Structure beyond word-level(!).
\end{itemize} 

\section{Word classes}

\ezimeti{
\item Why do we need word classes?
\item The reason is the same as why we need to categorize morphemes.
\item Word classes has three bases: 
\ex.
\a.		 morphosyntactic: the same morphological processes are applicable to
		to the members of a class;
\b.  distributional: the members of a class are interchangeable in a
		given slot of a sentence or a phrase (what we mean by ``phrase'' will
		come below).
\b. functional: the members of a class have the same type of
		contribution to the meaning of a sentence or phrase.


\item Most basic distinction is \uterm{lexical} versus \uterm{functional}
categories. 
	\item[] Lexical categories are rich in semantic content. E.g.\ noun, verb,
	adjective, adverb.
	\item[] Functional categories carry grammatical information that pertains to
	the organization of how expressions come together and how their meanings are
	composed to arrive at more complex meanings. E.g.\ prepositions,
	conjunctions, articles, and so on. 

\item Another distinction is between \uterm{closed-class}  versus
\uterm{open-class} words.

\item In most of the cases lexical categories are open, and functional
categories are closed; but there are exceptions.

}

\subsection{Verbs}

\ezimeti{

\item {\bf Function:} predication of a state, event, situation,
process, and so on. A general expression that is proposed to cover all these
types of objects is \uterm{eventuality}.

\item {\bf Distribution:} Verbs come in subclasses that determine their
distribution. 
	\ezimeti{
		\item Intransitive verbs have a single \uterm{argument}.
		\exg. Bh\'eic s\'e. \\
			sneezed he\\
			`He sneezed.'\hfill (Irish)

		\ex. \"O\u grenciler hap\c surdu(lar).

	
		\item Transitive verbs have two,


		\exg. Bhris s\'i an chathaoir. \\
			break.PAST she the chair\\
			`She broke the chair.'\hfill (Irish)
		
		\item Ditransitive verbs have three.

		\exg. human rassal-o makt$\bar{\text{u}}$b  le \textipa{P}ab$\bar{\text{u}}$-hum \\
			they send.PAST-3PL letter to father-their\\
`They sent a letter to their father.' \hfill (Chadian Arabic)


\item[] Some ditransitives have a transitive usage as well,  e.g.\ \emph{buy},
\emph{sell}, but not \emph{hand}.	

\item[] And some verbs can be both intransitive and tranitive, e.g.\
\emph{cook}, \emph{sing}, and so on.
	} 


\item {\bf Morphosyntax:} 
\ezimeti{
	\item \uterm{Tense} locates an eventuality in time; 

	\ex.
	\a.  {\bf ga}-\v ciux `He did it some time ago.'  \hfill (Chinook)
	\b. {\bf ni}-číux `He did it long ago.' 
	\b. {\bf na}-čiúx$^\text{w}$-a `He did it recently.'
	\b. {\bf i}-číux  `He just did it.'


	\item \uterm{Aspect} specifies the extent of an eventuality, e.g.\ whether
	it is completed, continuing, iterating, and so on; 

	\ex. 
	\a. ba-{\bf léé}-bomba  `They are working.’ (progressive)
    \b. ba-{\bf là}-bomba `They repeatedly work.’ (habitual)

	\item \uterm{Mood} indicates actuality, possibility, probability, certainty,
	and so on, of a proposition.  These categories are expressed with \uterm{modal auxiliaries} in
	Germanic languages (English, German\ldots). They are expressed in verbal
	morphology in Turkish and many other languages. Some people include the category of
	\uterm{evidentiality} -- marking the indirect knowledge of a proposition -- under mood as well. 


	\item \uterm{Valency changing processes} like \uterm{passive},
	\uterm{causative} and \uterm{reflexive} are usually indicated in verbal morphology.

	\ex. 
	\a. Adam k\"ope\u gi  y\i{}kad\i{}.
	\b. Adam y\i{}kand\i{}. \hfill (reflexive or passive)		
 	\b. Adam  k\"ope\u gi y\i{}katt\i{}. \hfill (causative)
	\b. *Adam k\"ope\u gi y\i{katt}\i{r}\i{ld}\i.
	\b. Adam  k\"ope\u ge y\i{kat}\i{ld}\i.
	\b. Adam k\"ope\u ge y\i{katt}\i{r}\i{ld}\i. \hfill (?)


	\item \uterm{Agreement} morphology on the verb marks the person, gender
	and/or  number of some (or all) of the arguments. 
	
		\exg. Nyuna na-tinu-nya na lau\\
		she 3SG.SU-weave-3SG.OBJ the sarong\\
		`She weaves the sarong.'\hfill (Kambera)


	}
}


\subsection{Nouns}
\ezimeti{
\item {\bf Function:} Noun phrases provide the arguments of a verb, specifying the
\uterm{semantic roles} in the eventuality expressed by the verb.

% \begin{uexercise}
% 
% Match the nouns in the sentences with the roles in the table. 
% 
% 
% \fbox{
% \begin{tabular}{lllll}
% experiencer &theme      &   stimulus & patient &  instrument \\
% experiencer  &recipient &   stimulus & patient &  agent 
% \end{tabular}
% }
% \end{uexercise}

\ex. 
\ag. Lee handed {the letter} to Kim. \\
	\textsc{Agent}  {\ } \textsc{Theme} {\ }  \textsc{Recepient} \\
\bg. Kim detests sprouts. \\
\textsc{Experiencer} {\ } \textsc{Stimulus}\\
\bg. Spiders frighten Lill.\\
\textsc{Stimulus} {\ } \textsc{Experiencer}\\
\bg. The flowers wilted.\\
\textsc{Patient} \\
\bg. {The ball} broke {the window}.\\
\textsc{Instrument} {\ } \textsc{Patient}\\




\item {\bf Distribution:} Every noun phrase appears in one of a number of
\uterm{grammatical relations}, like \uterm{subject}, \uterm{direct object},
\uterm{indirect object}, and so on.
 
\item[] In English you can spot subjects  by the position of the noun phrase, by
looking at which argument the verb agrees with, and so on.
% and from the form of the
% pronouns when the verb or auxiliary is finite -- we will come to what is meant
% by ``finite''.


\ex. 
\a. This woman buys all the best apples.
\b. All those people are enjoying our apples.
\b. Apples were grown in that orchard.
\b. Apples, she really enjoys.



\item It is important to note that the relation between grammatical relations
and semantic roles is quite complex.
% -- we will return to this issue.

\item {\bf Morphology:} Nouns and noun phrases in most languages carry a
selection of these grammatical categories:

\ezimeti{
\item \uterm{number};
\item \uterm{gender} (or class), which may have no semantic correlate;
\item \uterm{case} is a rather direct indicator of grammatical relations (Tr.\ ``ismin
halleri''). But not every case signals the same grammatical relation in every
language that has it (E.g.\ Icelandic has accusative subjects).
}

\item Another word class that is closely related to nouns and noun phrases is
\uterm{determiners}, which contains articles (\emph{the paper}, \emph{a cat},
demonstratives (\emph{this cat}),
\emph{wh}-determiners (\emph{which cat}), quantifiers (\emph{every cat},
\emph{some cat(s)}), possessive determiners (\emph{my cat}), determiner
pronouuns (\emph{we cats}, \emph{you linguists}).


} 
\subsection{Adjectives}

\ezimeti{
\item {\bf Function:}  Adjectives can be \uterm{predicative} or
\uterm{attributive}.
\item[] Attributive adjectives directly modify a noun.
\item[] Predicative adjectives occur in contexts like the following:

\ex.
\a. He felt \cntx.
\b. She seems \cntx.
\b. I find it \cntx to think she's an acrobat.

\item {\bf Distribution:}  Adjectives can be modified with \uterm{intensifiers}
like  \emph{very}, \emph{quite}, \emph{too}, \emph{rather}, \emph{somewhat},
\emph{enough}. 

\item {\bf Morphology:}  Adjectives can appear in \uterm{comparative} and
\uterm{superlative} forms.


\begin{uexercise}

Does Turkish have distinct classes of nouns and adjectives?

\end{uexercise}

}

\subsection{Adverbs}

\ex. 
\a. This [strangely sad] song.
\b. She spoke [strangely lucidly]. 
\b. She spoke strangely.


\ezimeti{
\item Functionally, adverbs modify the eventuality, rather than its
participants.
\item Adverbs do not differ from adjectives in form in some languages.
}

\subsection{Adpositions}
\ezimeti{
\item Adpositions mainly code information on time, location and manner. 
\item Some languages put adpositions before their arguments
(\uterm{prepositions}) and some put them after (\uterm{postpositions}).  
}

\section{Sentence structure}

\ezimeti{

\item The \uterm{clause} is a technical linguistic abstraction based on the
classical notion of a sentence, according to which a sentence is the result of
bringing together a subject and a predicate. 

\item There are different sub-types of clauses. Those where the verbal element
has or lacks grammatical information like tense, agreement, etc., those with or
without subjects, and so on. We will not be concerned with sub-typing of
clauses in detail.

\item A sentence is a structural organization of one or more clauses. A
\uterm{simplex} sentence has one clause, while a \uterm{complex} sentence has
more.



}

\subsection{Coordination and subordination}

\ezimeti{
\item The structurally simplest way of forming a complex clause is to
\uterm{coordinate} (or \uterm{conjoin}) two clauses by coordinating
conjunctions like the English \emph{and}, 
\emph{but}, or \emph{or}. Note that putting a
logical disjunction like \emph{or} is also conjoining.

\item In coordinating conjunctions, the conjuncts are syntactically on equal
terms. 
\item[] For instance, modulo pragmatic concerns, the order of the conjuncts does
not affect grammaticality.


\item In \uterm{subordination} of two clauses, there exists a structural
asymmetry between the clauses. 

\item[] {\bf Complement clauses:}

\ex.
\a. My friend claimed [(that) Ceri liked chips].
\b. I wondered [whether/if Lee had gone].
\b. They want [to leave before breakfast].
\b. [That Chris liked Lee so much] really surprises me.
\b. [For Mel to act so recklessly] shocked everyone.

\item One of the clauses (\uterm{complement clause}) appears as the complement
(or the argument) of the verb of another clause (\uterm{matrix} or \uterm{root} clause). Another
term for the relation is that the complement clause is \uterm{embedded} under
another clause.

\item Having clauses as components of other clauses is a typical instance of
\uterm{recursion} in natural languages, which is one source of structural
\uterm{hierarchy}. Theoretically three is no limit to the number of \uterm{levels of embedding}.

\ex. They want [to know [whether we’d expect [to leave before breakfast]]].

\item When we come to the topic of constituency, we will have a more solid
foundation for positing a hierarchical syntactic structure. 


\item[] {\bf Adjunct (or adverbial) clauses:}

\ex. 
\a. Mel will be there [when she’s good and ready].
\b. [If you’re leaving early], please get up quietly.
\b. [Kim having left early], we drank her beer.


\item These are optional subordinations; they give extra information, but not
required for a syntactically complete sentence.

% \item Certain grammatical processes are sensitive to matrix versus
% subordinate clause distinction. 
% \item[] One example is basic word order (German, Breton). 
% 
% \item[] Another is the subject/auxiliary inversion and tag questions in English.

}
\newpage
\section{Constituency and phrase structure}
\subsection{Motivation for constituent structure}

\ezimeti{
\item {\bf Motivation 1:} A primary indication of the fact that linguistic expressions are structured into
constituents is the notion of structural ambiguity.

\ex.\label{glob} I saw the man with the telescope.

\ex. \label{loc} Because the man that the police interrogated didn't call his lawyer was concerned.

\item Example \xref{glob} is \uterm{globally} ambiguous, while \xref{loc} is
\uterm{locally} ambiguous.

\item In \xref{glob} each \uterm{reading} corresponds to a different
\uterm{constituent structure}.

\item Here are the two structures for \xref{glob}:
\ezimeti{
\item[S1:]

\Tree [.$X_1$  I [.$X_2$ saw [.$X_3$ \edge[roof]; {the man with the telescope} ]]]

\item[S2:]

\Tree [.$X_1$  I [.$X_2$  [.$X_3$ saw [.$X_5$ \edge[roof]; {the man} ]]
 [.$X_4$ \edge[roof]; {with the telescope} ] ]]
}

\item As can be seen in the \emph{telescope} example, constituency is not a
property of strings alone. It makes sense to talk about constituency only in the
context of a structural description, which represents a certain interpretation
of the given expression.

\item {\bf Motivation 2:} Another motivation for constituency is that certain
syntactic operations and constraints refer to constituents; in other words,
there are generalizations that are best expressed in terms of constituents.
We will see some of these below when we discuss constituency tests. 
}

\subsection{Constituents and phrases}

\ezimeti{

\item Basics of trees:
	\ezimeti{
	\item a node $A$ dominates a node $B$ iff either $A$ is
		$B$ or there is a path from $A$ to $B$, such that moving along this path
		increases your distance from the root.
	\item the distance from a node $A$ to $B$ is the number of intervening edges
	along the shortest path from $A$ to $B$.
	\item root is the node that dominates every other node but is not dominated
	by any.
	\item two types of nodes: terminal and non-terminal.
	}

\item A maximally large sequence of terminals dominated 
by a single node is a constituent.

\item Just like morphological rules need to know word classes like verb, noun,
and so on, syntactic rules need to know phrase classes like verb phrase, noun
phrase, prepositional phrase, and so on.

\item A \uterm{phrase} is a structural organization of a \uterm{head}, it's
\uterm{complements} (or \uterm{dependents}), and a number of optional
\uterm{modifiers} (or \uterm{adjuncts}).

\item[] Here is a noun phrase first without, then with adjectival modification -- let's take the determiner to be a modifier of
the noun:


\parbox[t]{.1\textwidth}{
\Tree [.NP [.Det the ]
			[.N book ] ]

} 
\parbox[t]{.1\textwidth}{
\Tree [.NP [.Det the ]
		   [.N 
				[.Adj blue ]	
				[.N book ] 
		   ]
	  ]
}

\item[] Now a verb phrase, again without and with modification:

\parbox[t]{.17\textwidth}{
\Tree [.VP [.V read ]
	  		[.NP [.Det the ]
		   		[.N 
					[.Adj blue ]	
					[.N book ] 
			   ]
		    ]
	]
}
\parbox[t]{.1\textwidth}{
\Tree [.VP 
		[.VP [.V read ]
	  		[.NP [.Det the ]
		   		[.N 
					[.Adj blue ]	
					[.N book ] 
			   ]
		    ]
		]
		[.Adv quickly ]
	]
}

\item The above trees simplify the structure by treating Adj and Adv as
non-phrasal. But they can have modifiers and complements as well. 

\Tree [.VP 
		[.VP [.V read ]
	  		[.NP [.Det the ]
		   		[.N 
					[.AdjP [.Adv faintly ] [.Adj blue ] ] 
					[.N book ] 
			   ]
		    ]
		]
		[.AdvP [.Adv extremely ] [.Adv quickly ] ]
	]
}

\subsection{Constituency tests}
\ezimeti{
\item[] {\bf Sentence fragment test:}

\item[] Any expression that can be put as an answer to a \emph{wh}-question
(=a question involving \emph{what}, \emph{which}, \emph{who}, and so on) is a
constituent.

\ex.
\a. Kim wrote that book with the blue cover.
\b. Kim bought that book with her first wages.

\ex.
\a. What did Kim write?
\b. That book with the blue cover.

\ex.
\a. What did Kim buy?
\b. *that book with her first wages. 

\item[] {\bf Echo question test:}

\ex.
\a. *Kim wrote {\bf what} with the blue cover?
\b. Kim bought {\bf what} with her first wages?

\ex. Kim wrote that {\bf what} with the blue cover?


\item[] {\bf The cleft test:}

\ex. 
\a. Kim wrote that book with the blue cover.
\b. Kim bought that book with her first wages.

\ex.
\a. It was [that book with the blue cover] that Kim wrote
\b.  *It was [that book with her first wages] that Kim bought.

\item[] {\bf Ellipsis test:}

\ex.
\a. John wrote a book last year. Jane did so (too).
\b. John wrote a book last year. Jane did so the year before.


\item[] What can the following sentence mean?

\ex. John saw the man with the telescope, Jane did so with the monocular.

\ex. 
\a. John sent a book to Jane. Irma did so to Harry.
\b. *John asked Bill to send a book to Jane. Irma did so to Harry.

\ex.
\a. Ali hızlı araba kullanıyor; Mehmet yavaş.
\b. ??Ali hızlı araba kullanıyor; Mehmet motosiklet.

\ex.
\a. Ahmet Veli'yi Aylin'e şikayet etii; Hasan da Sevgi'yi öyle.
\b. ??Ahmet Veli'yi Aylin'e şikayet etii; Hasan da Bilge'ye öyle.

\begin{uexercise}
Work out the constituent structure of the following sentences by
constituency tests:
\begin{enumerate}
\item My brother wrote down his address.
\item My brother applied for this job.
\item John expected Jane to leave.
\item John promised Jane to leave. 
\end{enumerate}
\end{uexercise}

\item[] {\bf Coordination test:}

\item[] Only constituents having the same syntactic category can get combined via coordinating conjunctions.
}

% \section*{Self study}
% 
% The material we have covered and some more can be reviewed by reading Chapters 2, 3,
% 5 of \ctnm{tallerman11}. 


\bibliography{ozge}
\bibliographystyle{apalike}
\end{document}


\section{Heads and Dependents}

Tie this to configuration through order and hierarchy -- also tie this to why we
do not simply and only have semantics.


\section{Heads and Dependents}

Three types Tallerman p. 111:
\ex.
\a. the selection of a specific type of argument by a head
\b. agreement: the copying of features froma head to its dependents
\b. government by a head




\section{Encoding dependency: case, agreement, configuration}

Start with Turkish, then German, than English/German/Turkish for agreement.

State that there are languages that have more than subject agreement, and there
are languages that have no agreement. 

Tallerman agreement example from the language where agreement depends on the
word order. The L is basically SVO, where the verb agrees only with the subject;
but when the order becomes SOV or OSV, then there is object agreement as well.

Tallerman p. 210 Icelandic excercise is very nice, there is an agreement
triggered by a PRO subject in the embedded clause. Icelandic is also
interesting in explaining that case-semantics connection is not
straightforward.

You can use Tagalog (Tallerman 196-7) to show how creative linguistic system can
be.


\section{Semantic Roles}

First adapt the list on Tallerman p.46 to give the notion of semantic roles.


\section{Grammatical Relations}

Structural indicators: case, agreement, configuration (order and hierarchy).

Case:

Talk about the relation between case and word-order flexibility.


\section{Syntax-Semantics interface}

Semantics/Syntax relation is not direct -- use theta roles on p. 46 of Tallerman
to motivate this.

Activity: compile grammar from Google translate, either for a given fragment, or
a fragment of your choice.

Tallerman p.237 complex (non-compositional) person agreement morphemes. 

\bibliography{ozge}
\bibliographystyle{apalike}
\end{document}
